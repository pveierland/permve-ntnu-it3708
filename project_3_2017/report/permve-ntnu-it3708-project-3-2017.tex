\documentclass[paper=a4, fontsize=9pt]{scrartcl}
\usepackage[bottom=0.7in, left=0.5in, right=0.5in, top=0.75in, foot=0.35in]{geometry}
\usepackage{layouts}

\usepackage[usenames,dvipsnames,x11names]{xcolor}

\usepackage[T1]{fontenc}
\usepackage{fourier}
\usepackage[english]{babel}
\usepackage{amsmath,amsfonts,amsthm}

\usepackage{sectsty}
\allsectionsfont{\centering \normalfont\scshape}

\usepackage{acronym}
\usepackage{booktabs}
\usepackage{caption}
\usepackage{fancyhdr}
\usepackage{float}
\usepackage{graphicx}
\usepackage[htt]{hyphenat}
\usepackage{lastpage}
\usepackage{listings}
\usepackage{multicol}
\usepackage{titlesec}
\usepackage[inline]{enumitem}
\usepackage{algorithm, algpseudocode}

\pagestyle{fancyplain}
\fancyhead{}
\fancyfoot[L]{}
\fancyfoot[C]{\thepage~of~2}
\renewcommand{\headrulewidth}{0pt}
\renewcommand{\footrulewidth}{0pt}
\setlength{\headheight}{13.6pt}

\newcommand{\horrule}[1]{\rule{\linewidth}{#1}}

\title{
\vspace{-1cm}
\normalfont \normalsize
\textsc{Norwegian University of Science and Technology\\IT3708 -- Bio-Inspired Artificial Intelligence}
\horrule{0.5pt} \\[0cm]
\Huge Project 3: Segmentation of Color Image\\Using Multi-Objective Evaluation Algorithm\\[-0.3cm]
\horrule{2pt} \\[0.1cm]
}

\author{Per Magnus Veierland\\permve@stud.ntnu.no}

\date{\normalsize\today}

\begin{document}

\maketitle

\begin{multicols}{2}

\section*{Image segmentation}

% How does your MOEA find segmentation? (0.5p)

\section*{MOEA Implementation}

% Describe  your  implementation,  only  for  MOEA  optimizing  three  objectives.  Your  description should  include  every  step  of  your  implementation  including  representation  and  evolutionary operators. Using figure(s) for chromosome is a must. (2p)

\section*{Parameters}

% Mention the values for every parameter of your chosen MOEA. (0.5p) 

\section*{Solutions}

% For any of the test images, provide one of the segmentation solution from the final optimal/near-optimal (for single objective) and Pareto-optimal (for two and three objectives) solutions. You need to use the same test image for all categories (3 single objective, 3 two objectives, 1 three objectives). You need to show both types (as Fig. 2a and 2b) of segmentation for each solution.   (1p)

% (3 + 1) objectives x 3 types of solutions = 12 images
% Include objective values + segmentation count + image ID for each

\section*{Pareto fronts}

% Plot Pareto-fronts (3 two objectives, 1 three objectives) for the same solutions those you will use in Point-3.  (1p)  

% (3 + 1) objectives = 4 plots

\bibliographystyle{unsrt}
\bibliography{references.bib}

\end{multicols}

\end{document}
