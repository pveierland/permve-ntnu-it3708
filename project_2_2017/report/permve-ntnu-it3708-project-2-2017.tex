\documentclass[paper=a4, fontsize=10pt]{scrartcl}
\usepackage[bottom=0.8in, left=0.5in, right=0.5in, top=0.8in, foot=0.4in]{geometry}
\usepackage{layouts}

\usepackage[usenames,dvipsnames,x11names]{xcolor}

\usepackage[T1]{fontenc}
\usepackage{fourier}
\usepackage[english]{babel}
\usepackage{amsmath,amsfonts,amsthm}

\usepackage{sectsty}
\allsectionsfont{\centering \normalfont\scshape}

\usepackage{acronym}
\usepackage{booktabs}
\usepackage{caption}
\usepackage{fancyhdr}
\usepackage{float}
\usepackage{graphicx}
\usepackage[htt]{hyphenat}
\usepackage{lastpage}
\usepackage{listings}
\usepackage{longtable}
\usepackage{minted}
\usepackage{multicol}
\usepackage{multirow}

\pagestyle{fancyplain}
\fancyhead{}
\fancyfoot[L]{}
\fancyfoot[C]{\thepage~of~2}
\renewcommand{\headrulewidth}{0pt}
\renewcommand{\footrulewidth}{0pt}
\setlength{\headheight}{13.6pt}

\newcommand{\horrule}[1]{\rule{\linewidth}{#1}}

\usepackage{tikz}
\usetikzlibrary{plotmarks}
\usepackage{pgfplots, pgfplotstable}
\pgfplotsset{compat=1.5}
\usepgfplotslibrary{colorbrewer}
\usepgfplotslibrary{fillbetween}
\usepgfplotslibrary{groupplots}

% http://tex.stackexchange.com/questions/67895/is-there-an-easy-way-of-using-line-thickness-as-error-indicator-in-a-plot

% Takes six arguments: data table name, x column, y column, error column,
% color and error bar opacity.
% ---
% Creates invisible plots for the upper and lower boundaries of the error,
% and names them. Then uses fill between to fill between the named upper and
% lower error boundaries. All these plots are forgotten so that they are not
% included in the legend. Finally, plots the y column above the error band.
\newcommand{\errorband}[6]{
\pgfplotstableread{#1}\datatable
  \addplot [name path=pluserror,draw=none,no markers,forget plot]
    table [x={#2},y expr=\thisrow{#3}+\thisrow{#4}] {\datatable};

  \addplot [name path=minuserror,draw=none,no markers,forget plot]
    table [x={#2},y expr=\thisrow{#3}-\thisrow{#4}] {\datatable};

  \addplot [forget plot,fill=#5,opacity=#6]
    fill between[on layer={},of=pluserror and minuserror];

  \addplot [#5,thick,no markers]
    table [x={#2},y={#3}] {\datatable};
}

\newacro{GA}{Genetic Algorithm}
\newacro{MDVRP}{Multi-Depot Vehicle Routing Problem}

\title{
\vspace{-1cm}
\normalfont \normalsize
\textsc{Norwegian University of Science and Technology\\IT3708 -- Bio-Inspired Artificial Intelligence}
\horrule{0.5pt} \\[0cm]
\huge Project 2:\\ Solving the \ac{MDVRP} using \acp{GA} \\[-0.3cm]
\horrule{2pt} \\[0.1cm]
}

\author{Per Magnus Veierland\\permve@stud.ntnu.no}

\date{\normalsize\today}

\begin{document}

\maketitle

\begin{multicols}{2}

\section*{Chromosome representation}

\section*{Crossover operator}

\section*{Mutation operator}

\section*{Selection mechanism}

\section*{Parameters}

\section*{Feasibility}

%Describe whether or not the crossover and mutation operators will produce infeasible off-spring(s) after executing. If yes, how did you handle that? If not, why? 

\section*{Fitness function}

\section*{Elitism}

% Describe whether or not you implemented elitism. If yes, how? If not, why? (0.5p) 

\section*{Solution}

% Using the parameter values mentioned in (b), present the solution of any one of the test problems. The solutions should be presented as (i) graphically, and (ii) in the provided solution data file format.  You  should  follow  the  solution presentation  guidelines  mentioned  earlier.  You  must present the solutions using screen shots from your computer. (1p) 

% TODO fitness function use min or use sum

% TODO feasible vs infeasible. Keep in population and track feasibility? Use for diversity.

% TODO consider plotting routing without lines to depots for clarity

% If broken individuals, crossover operation will tend to fix on average. Can track number of broken at a time in population. Can use best non-broken individual. Elitism should take brokenness into account

\end{multicols}

\end{document}
