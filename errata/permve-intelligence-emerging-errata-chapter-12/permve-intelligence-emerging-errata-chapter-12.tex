\input{permve-ntnu-latex-assignment.tex}

\title{
\normalfont \normalsize
\textsc{Norwegian University of Science and Technology\\IT3708 -- Subsymbolic Methods in AI\vspace{-0.2cm}}
}

\author{Per Magnus Veierland\\permve@stud.ntnu.no}

\date{\normalsize\today}

\hyphenation{LeftParent}
\hyphenation{RightParent}
\hyphenation{LeftChild}
\hyphenation{RightChild}

\begin{document}

\fancyfoot[C]{}
\maketitle

\vspace{-1.3cm}
\section*{Intelligence Emerging: Chapter 12 Errata Suggestion}
\vspace{-0.15cm}

\begin{enumerate}
\item {\color{WildStrawberry} The lower two neural network diagrams (after crossover) shown in Figure~12.4 on page~321 appear to be incorrect.}

Before the crossover there are two individuals; \textit{LeftParent} and \textit{RightParent}. \textit{LeftParent} has the genotype \framebox{\texttt{1-11~-111}}, describing an excitatory connection from \textbf{X} to \textbf{A}, an inhibitory connection from \textbf{Y} to \textbf{A}, an excitatory connection from \textbf{A} to \textbf{C}, an inhibitory connection from \textbf{X} to \textbf{B}, an excitatory connection from \textbf{Y} to \textbf{B}, and an excitatory connection from \textbf{B} to \textbf{C}. {\color{ForestGreen} The genotypes for \textit{LeftParent} and \textit{RightParent} seems to correspond with both their networks diagrams and their equivalent logic descriptions.}

During crossover, the \textit{LeftParent} genotype \framebox{\texttt{1-11~-111}} is crossed with the \textit{RightParent} genotype \framebox{\texttt{-111~1-11}} to produce \textit{LeftChild} with the genotype \framebox{\texttt{1-11~1-11}} and \textit{RightChild} with the genotype \framebox{\texttt{-111~-111}}. {\color{ForestGreen} The children genotypes seem to be the correct result from the crossover.}

However, the children genotypes do not seem to correspond with their network diagrams. The genotype of \textit{LeftChild} describes an excitatory connection from \textbf{X} to \textbf{A}, which in the diagram is an inhibitory connection, and the genotype describes an inhibitory connection from \textbf{Y} to \textbf{B}, which in the diagram is an excitatory connection. Equally for the \textit{RightChild} where the genotype describes an inhibitory connection from \textbf{X} to \textbf{A}, which in the diagram is an excitatory connection, and the genotype describes an excitatory connection from \textbf{Y} to \textbf{B}, which in the diagram is an inhibitory connection.

\underline{Conclusion}: {\color{ForestGreen} The child genotypes describes networks which correctly corresponds to their noted equivalent logic description $X \land \neg Y$ and $Y \land \neg X$.} {\color{WildStrawberry} The shown child network diagrams both has the equivalent logic description $X \lor Y$, which does not correspond to the genotype or noted equivalent logic description.}

\item For readability a note could be added to Figure~12.5 on page~322 which explains that the node indexes used within genomes must be read as modulo the number of input/output nodes, given the node type.

\item In Figure~12.6 on page~325 there seems to be a typo where ``\textsc{XOR(X),Y}'' and ``\textsc{OR(X),Y}'' should read ``\textsc{XOR(X,Y)}'' and ``\textsc{OR(X,Y)}''. There also seems to be two different notations used within the same figure; ``\textsc{OR(X,Y)}'' and ``\textsc{X or Y}''. In the bottom right network there seems to be a missing asterisk for the right-most ``B''-node.
\end{enumerate}

\end{document}

