\documentclass[paper=a4, fontsize=10pt]{scrartcl}
\usepackage[bottom=1in, left=1in, right=1in, top=1in]{geometry}
\usepackage{layouts}

\usepackage[usenames,dvipsnames,x11names]{xcolor}

\usepackage[T1]{fontenc}
\usepackage{fourier}
\usepackage[english]{babel}
\usepackage{amsmath,amsfonts,amsthm}

\usepackage{sectsty}
\allsectionsfont{\centering \normalfont\scshape}

\usepackage{tikz}
\usepackage{pgfplots}
\usetikzlibrary{plotmarks}

\usepackage{booktabs}
\usepackage{longtable}
\usepackage{tabularx}
\usepackage{ragged2e}
\newcolumntype{Y}{>{\RaggedRight\arraybackslash}X}
\usepackage{paralist}

\usepackage{acronym}
\usepackage[inline]{enumitem}
\usepackage{fancyhdr}
\usepackage{graphicx}
\usepackage{lastpage}
\usepackage{listings}
\usepackage{multirow}
\usepackage{subfigure}

\pagestyle{fancyplain}
\fancyhead{}
\fancyfoot[L]{}
\fancyfoot[C]{\thepage~of~4}
\renewcommand{\headrulewidth}{0pt}
\renewcommand{\footrulewidth}{0pt}
\setlength{\headheight}{13.6pt}

\newcommand{\horrule}[1]{\rule{\linewidth}{#1}}

% For code fragments:
%\usepackage{fontspec}
\usepackage{minted}
%\setsansfont{Calibri}
%\setmonofont{Consolas}

\usepackage{algorithm, algpseudocode}
\usepackage{caption}
\usepackage{float}
\usepackage{tabu}

\usepackage{hyperref}
\hypersetup{hidelinks}

\pgfplotsset{compat=1.5}
\usepackage{pgfplots, pgfplotstable}
\usepgfplotslibrary{colorbrewer}
\usepgfplotslibrary{fillbetween}

\pgfplotsset{
every axis/.append style={
scale only axis,
width=0.40\textwidth,height=0.3\textwidth,
},
/tikz/every picture/.append style={
trim axis left,
trim axis right,
baseline
}
}

% http://tex.stackexchange.com/questions/67895/is-there-an-easy-way-of-using-line-thickness-as-error-indicator-in-a-plot

% Takes six arguments: data table name, x column, y column, error column,
% color and error bar opacity.
% ---
% Creates invisible plots for the upper and lower boundaries of the error,
% and names them. Then uses fill between to fill between the named upper and
% lower error boundaries. All these plots are forgotten so that they are not
% included in the legend. Finally, plots the y column above the error band.
\newcommand{\errorband}[6]{
\pgfplotstableread{#1}\datatable
  \addplot [name path=pluserror,draw=none,no markers,forget plot]
    table [x={#2},y expr=\thisrow{#3}+\thisrow{#4}] {\datatable};

  \addplot [name path=minuserror,draw=none,no markers,forget plot]
    table [x={#2},y expr=\thisrow{#3}-\thisrow{#4}] {\datatable};

  \addplot [forget plot,fill=#5,opacity=#6]
    fill between[on layer={},of=pluserror and minuserror];

  \addplot [#5,thick,no markers]
    table [x={#2},y={#3}] {\datatable};
}


\title{
\normalfont \normalsize
\textsc{Norwegian University of Science and Technology\\IT3708 -- Bio-Inspired Artificial Intelligence}
\horrule{0.5pt} \\[0.4cm]
\huge Project 1:\\ Supervised and Reinforcement Learning of\\Neural Agent Controllers\\
\horrule{2pt} \\[0.5cm]
}

\author{Per Magnus Veierland\\permve@stud.ntnu.no}

\date{\normalsize\today}

\begin{document}

%\fancyfoot[C]{}
\maketitle

\section*{Genotype}


\renewcommand{\theFancyVerbLine}{
  \sffamily\textcolor[rgb]{0.5,0.5,0.5}{\scriptsize\arabic{FancyVerbLine}}}

\begin{figure}[H]
\begin{minted}[mathescape,
               linenos,
               numbersep=5pt,
               frame=lines,
               framesep=2mm]{python}
def train(self, percepts, target_action, learning_rate):
  inputs   = encode_percepts(percepts)
  outputs  = np.dot(self.weights, inputs)
  outputs -= np.max(outputs) # Shift values for softmax numerical stability
  softmax  = np.exp(outputs) / np.sum(np.exp(outputs))
  one_hot_target = np.zeros(3)
  one_hot_target[target_action] = 1 
  delta = one_hot_target - softmax
  self.weights += learning_rate * np.dot(delta.reshape((3, 1)), inputs.reshape((1, 12)))
\end{minted}
\vspace*{-5mm}
\caption{\textit{Widroff-Hoff} rule implementation with cross-entropy ``softmax'' loss.}
\end{figure}

\begin{figure}[H]
\centering
\begin{tabularx}{\textwidth}{XcXc}
~ &
\begin{tikzpicture}
\begin{axis}[xlabel={Generations},ylabel={Fitness / time step}]
\errorband{../data/learning-curve-supervised-25-training-rounds-10-average.txt}{0}{1}{2}{Cyan}{0.4}
\end{axis}
\end{tikzpicture}
& ~ &
\begin{tikzpicture}
\begin{axis}[xlabel={Generations},ylabel={Fitness / time step}]
\errorband{../data/learning-curve-supervised-25-training-rounds-10-average.txt}{0}{1}{2}{Cyan}{0.4}
\end{axis}
\end{tikzpicture}
\\
\end{tabularx}
\caption{\ac{EANN} performance when trained on 1~static~scenario~(left), and on 5~static~scenarios~(right). Mean population fitness is shown in blue with standard deviation shown in light blue, and the fitness of the best individual is shown in red.}
\label{fig:performance_static}
\end{figure}

\end{document}

